\section{Dinic 1970}
\label{DinicSection}

The Dinic algorithm was published in 1970 by Yefim Dinic \cite{dinic1970}. It is the paper that introduced the level graph and blocking flow, 
which is basically a way of reducing the size of the graph before looking for augmenting paths.
The running time of this algorithm is $O(n^2m)$ which should make it perform better on dense graphs than the algorithm by Edmonds and Karp.

\subsection{The Algorithm}

The algorithm first does a breath first search to filter some of the edges. In the search it marks nodes according to their distance from $s$.
Only edges $(u, v)$ where the distance from $s$ to $u$ is less than the distance from $s$ to $v$ are used in the next step. 
Additionally, once $t$ has been reached, no edges $(u, v)$ should be added where the distance from $s$ to $v$ is greater than the distance from $s$ to $t$.
This results in a level graph that potentially has much fewer edges than the original graph.
The special property of this graph is that all paths that goes from $s$ to $t$, will have the same length $k$.
We then run a single depth first search on the graph to find all augmenting paths of length $k$. 
When the depth first search reaches $t$, we send the flow on the path like in the algorithm by Edmonds and Karp, 
jump back behind the first bounding edge on the path, and continue the depth first search from there.
This means that the entire part of the graph that is in front of the bounding edge can be visited again if there is another path going to it.
To avoid having the search process a part of the graph that can not reach $t$ multiple times, we mark nodes that can not reach $t$ as dead when they are discovered.
When we have visited all edges on a node, we know that that node can no longer reach $t$. Edges going to dead nodes are removed once discovered.
When the depth first search is done, and we have found all augmenting paths in the layer graph, we compute the residual network of the original graph.
A new layer graph can then be computed based on the residual network.
This process repeats until we arrive at a layer graph that is not connected to $t$, at which point all augmenting paths have been found.

Correctness follows from the same argument as in Section~\ref{EK1972Section}.
We always have a valid flow, and at the end of the algorithm, no augmenting path can be found from $s$ to $t$ in the residual network.


\subsection{Analysis}

Every time we have found a blocking flow in a level graph, we have found all augmenting paths of length $k$. The next level graph must have augmenting paths strictly longer than $k$.
The reason is the same as in Section~\ref{EK1972Analysis}. The distance to a node never decrease because the nodes in an augmenting path have increasing distance from $s$.
Instead of this being due to using a breath first search, it is because the level graph only contains edges to nodes that have a higher distance from $s$.
Since the distance to $t$ never decrease, and we push along all augmenting paths of length $k$, all subsequent augmenting paths must have a length greater than $k$.
The longest path possible from $s$ to $t$ is of length $n$, so we need to calculate level graphs and blocking flows in at most $n$ iterations. 

Every time the depth first search visits a node, it runs through all of its edges.
When it is done with one of the edges, and that part of the graph is dead, it removes the edge.
This means that edges to nodes that can not reach $t$ can only be used once. So discovering and trimming the dead parts of the graph takes $O(m)$ time in total.

If the part of the graph it visits is not dead, the path will end up by reaching $t$. In this case, an augmenting path is saturated, and the bounding edge is removed.
This path will have length $k$, which is at most $n$, so finding and saturating the path will take $O(n)$ time. 
Since we remove an edge every time we saturate a path, we can find at most $m$ such paths. This means that finding the paths take $O(nm)$ time.

Computing the level graph was done with a breath first search that stops when it reaches $t$.
A breath first search takes $O(m)$ time, so this yields the running time $O\left(n(m+nm)\right)=O(n^2m)$.

Dynamic trees can be utilized to find the blocking flow in $O(m\log n)$ time, reducing the running time to $O(nm\log n)$, 
but dynamic trees was not introduced until 1983 by D. Sleator and R. E. Tarjan \cite{sleator1983}.

