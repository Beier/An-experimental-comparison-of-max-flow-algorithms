
\section{Dynamic Trees} \label{Dynamic Trees}

Dynamic trees, also called link cut Trees, is a data structure that was presented by D. D. Sleator and R. E. Tarjan in 1983 \cite{sleator1983}.
It is a forest where each node in a tree can maintain set of values assigned to them, such as a cost.
Updates done to the tree are performed in amortized $O(\log k)$ time, where $k$ is the size of the tree the update is performed on.
The operations supported by dynamic trees are

\begin{description}
  \item[Link($a$, $b$)] \hfill \\
  Make $a$ a child of $b$. $a$ has to be a root for this to be possible.
  \item[Cut($a$)] \hfill \\
  Remove the edge from $a$ to its parent, making $a$ a root node in its own tree.
  \item[SetCost($a$, $c$)] \hfill \\
  Set the cost of $a$ to $c$.
  \item[GetCost($a$)] \hfill \\
  Returns the cost of $a$.
  \item[AddCost($a$, $c$)] \hfill \\
  Modify the cost of $a$ and all of its ancestors by adding $c$.
  \item[GetPathLength($a$)] \hfill \\
  Returns the number of nodes on the path from $a$ to the root of the tree $a$ is in.
  \item[GetRoot($a$)] \hfill \\
  Returns the root of the tree $a$ is in.
  \item[GetChildren($a$)] \hfill \\
  Returns the children of $a$.
  \item[GetMinCostNode($a$)] \hfill \\
  Returns the last node on the path from $a$ to the root of the tree that has the minimum cost of all the nodes on the path.
  \item[GetBoundingNode($a$, $c$)] \hfill \\
  Returns the first node on the path from $a$ to the root of the tree that has a cost less than or equal to $c$.
\end{description}

In max flow algorithms, a dynamic forest is typically used to represent paths of nodes in a way such that each node in the graph has a corresponding node in the dynamic forest.
The nodes in the graph will have an edge which is the preferred edge to move flow along. Linking $a$ to $b$ represents that the edge $(a, b)$ is the preferred edge of $a$.
The cost of the nodes in the dynamic forest will represent the residual capacity on the preferred edge.
This way, a \emph{Tree~Push} can be performed, where moving flow along a path can be done in $O(\log n)$ time instead of $O(n)$ time.
The specifics of how this is done depends on the max flow algorithm, the way it is done for Goldberg Tarjan in Section~\ref{GoldbergTarjanSection} is
\begin{align*}
&v_j \gets \mathrm{GetMinCostNode}(v_i)\\
&c \gets \mathrm{GetCost}(v_j)\\
& \mathrm{AddCost}(v_i, -c)\\
& \text{Cut all GetMinCostNodes with 0 cost}
\end{align*} 
Since the dynamic tree can update the residual capacity of $k$ edges in $\log k$ time, it is not possible to keep a value updated in the graph.
For this reason, algorithms will have an incorrect residual capacities in the graph while an edge or its corresponding back edge is in the dynamic tree. 
The residual capacities are instead updated when a node is cut from the tree.
