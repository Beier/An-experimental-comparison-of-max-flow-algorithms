
\section{Terminology}
\label{TerminologySection}

We use $G=(V, E)$ to symbolise the graph that we are running the max flow algorithms on.
Here $V$ is a set of nodes, $E$ is a set of edges, and $(u, v)\in E$ is a directed edge where $u\in V$, $v \in V$ and $u \not= v$.
We use $n$ and $m$ to symbolise the number of nodes and the number of edges in the graph, respectively.
If $(u, v)$ exists in $E$, we assume that $(v, u)$ also exists in $E$.
With the max flow problem, two nodes, \emph{source} and \emph{target} are given. We denote them by $s$ and $t$ respectively.

We assume without loss of generality that all nodes can be reached from the source, and that all nodes can reach the target. 
Graphs that do not satisfy this assumption can be trimmed to fit the constraint in $O(m)$ time by performing breadth first searches from the source and from the target.

A \emph{path} in a graph is defined as a list of nodes $(v_1, \ldots , v_k)$ where \\
$(v_i, v_{i+1}) \in E$ for $i=1, \ldots, k-1$ and the list contains no duplicates.

Every edge $(u, v)$ has a \emph{capacity} associated with it denoted by $cap(u, v)$. The capacity of an edge must be a non-negative integer. This is an upper bound on the amount of flow we are allowed to send on the edge.
We use $U$ to represent the maximum capacity over all edges in the graph.
The actual \emph{flow} sent on an edge is denoted by $f(u, v)$. As with capacity, the flow on an edge must be a non negative integer. 
\emph{Residual capacity} on an edge is the amount of flow that can still be sent on the edge without violating the capacity constraint.
It is defined as $r(u, v) = cap(u, v) - f(u, v) + f(v, u)$. For edges $(u, v)\not\in E$, we define $cap(u, v)=f(u, v)=r(u, v)=0$.
An edge $(u, v)$ is said to be \emph{saturated} if $r(u, v) = 0$, and the act of saturating an edge is changing the flow to make the edge become saturated.

A path $P=(v_1, \ldots , v_k)$ is said to be \emph{residual} if $r(v_i, v_{i+1}) > 0$ for $i=1, \ldots, k-1$.
An \emph{augmenting path} $P=(v_1, \ldots , v_k)$ is a residual path in $G$ where $v_1 = s$ and $v_k = t$.
In other words, an augmenting path is a path from $s$ to $t$ in the residual network, where it is possible to send more flow.
The \emph{bounding edges} of an augmenting path is the edges that have the minimum residual capacity of all edges in the path.
As a consequence, if additional flow equal to the minimum residual capacity was pushed on the path, these edges would become saturated.

The \emph{distance} between two nodes $u$ and $v$ is denoted $distance(u, v)$, and is the number of edges connecting nodes in the shortest residual path connecting the two nodes.
If no such path exists, $distance(u, v)=n$. 
This could be infinity instead of $n$, but since the longest path possible contain all nodes, there can at most be $n-1$ edges in such a path.
Infinity can not be represented in a traditional integer, so $n$ is easier to work with.


The \emph{excess} of a node $v$, $e(v)$ is how much flow has been sent to the node, but not sent on to other nodes. It is defined as $e(v)=\sum_{u \in V}\left(f(u, v) - f(v, u)\right)$. 
This may generally only be negative for the node $s$ and positive for the node $t$ and 0 for all other nodes.


In order to have a valid flow, the following conditions must be met:
\begin{enumerate}
  \item  $\forall v \in V \setminus \{s, t\}, e(v) = 0$ 
  \item $\forall (u, v) \in E , f(u, v) \leq cap(u, v)$
\end{enumerate}
The first is referred to as the flow conservation constraint, and the second is the capacity constraint.
Some algorithms works by manipulating \emph{preflow}, which is a flow in the graph where the excess of nodes are allowed to be positive, thus violating the flow conservation constraint.

Tables of all definitions used throughout the thesis can be found in Appendix~\ref{TerminologyAppendix}.